%% LyX 2.4.3 created this file.  For more info, see https://www.lyx.org/.
%% Do not edit unless you really know what you are doing.
\documentclass[10pt,english]{extarticle}
\usepackage{mathptmx}
\usepackage{avant}
\renewcommand{\familydefault}{\rmdefault}
\usepackage[T1]{fontenc}
\usepackage[latin9]{inputenc}
\setcounter{secnumdepth}{1}
\setcounter{tocdepth}{1}
\usepackage{array}
\usepackage{verbatim}
\usepackage{longtable}
\usepackage{booktabs}
\usepackage{url}
\usepackage{multirow}
\usepackage{tabularx}
\usepackage{xltabular}
\usepackage{amsmath}
\usepackage{amsthm}
\usepackage[a4paper]{geometry}
\geometry{verbose,tmargin=2cm,bmargin=2cm,lmargin=2cm,rmargin=2cm,headheight=1cm,headsep=1cm,footskip=1cm}
\usepackage{fancyhdr}
\pagestyle{fancy}
\PassOptionsToPackage{normalem}{ulem}
\usepackage{ulem}

\makeatletter

%%%%%%%%%%%%%%%%%%%%%%%%%%%%%% LyX specific LaTeX commands.
\providecommand{\LyX}{\ensureascii{%
  L\kern-.1667em\lower.25em\hbox{Y}\kern-.125emX\@}}
\newcommand{\noun}[1]{\textsc{#1}}
%% Because html converters don't know tabularnewline
\providecommand{\tabularnewline}{\\}

%%%%%%%%%%%%%%%%%%%%%%%%%%%%%% User specified LaTeX commands.
\usepackage{graphicx,titlesec,amsmath,enumitem,environ,longtable,advdate}

%\titlespacing{\section}{0pt}{-2pt}{-2pt}
%\titleformat{\section}{\vbox{\rule{\linewidth}{0.5pt}}\vspace{-0.3cm}\LARGE\bfseries}{\thesection}{}{}

\newcommand{\yesterday}{{\AdvanceDate[-1]\today}}
\setlist{noitemsep}


%\date{}

\makeatother

\usepackage{babel}
\begin{document}

\lhead{Lin JIU}

\rhead{\yesterday}

\chead{{\Large\textsc{Curriculum Vitae}}}

\noindent{\large\textbf{\textsc{Contact}}}{\large\par}

\noindent{}%
\noindent\begin{minipage}[t]{1\columnwidth}%
\begin{tabularx}{\columnwidth}{>{\raggedright\arraybackslash}Xr}
Assistant Professor of Mathematics & \emph{E-mails}:\hspace{0.02\columnwidth}lin.jiu@dukekunshan.edu.cn\tabularnewline
Duke Kunshan University & lin.jiu.work@gmail.com\tabularnewline
8 Duke Ave, Kunshan, Suzhou, & \emph{Tel}:\hfill{}+86-0512-36657333\tabularnewline
Jiangsu Province, China, 215316 & \emph{Website}:\hfill{}https://JiuLin90.github.io\tabularnewline
 & \tabularnewline
\end{tabularx}%
\end{minipage}

\noindent{\large\textbf{\textsc{Employment}}}{\large\par}

\noindent{}%
\noindent\begin{minipage}[t]{1\columnwidth}%
\begin{tabularx}{\columnwidth}{l>{\raggedright\arraybackslash}Xr}
2023.08\textendash{} & \multicolumn{2}{l}{Assistant Professor of Mathematics\hfill{}Duke Kunshan University}\tabularnewline
2023.08\textendash{} & \multicolumn{2}{l}{Assistant Professor of the Practice of DKU Studies\hfill{}Duke University}\tabularnewline
2024.07\textendash{} & \multicolumn{2}{l}{Adjunct of the Faculty of Graduate Studies\hfill{}Dalhousie University}\tabularnewline
2020.08\textendash 2023.07 & \multicolumn{2}{l}{Lecturer in Mathematics\hfill{}Duke Kunshan University}\tabularnewline
2020.08\textendash 2023.07 & \multicolumn{2}{l}{Assistant Professor of the Practice\hfill{}Trinity College of Arts
\& Science, Duke University}\tabularnewline
2019.09\textendash 2020.07 & \multicolumn{2}{l}{Research Associate\hfill{}Department of Mathematics and Statistics,
Dalhousie University}\tabularnewline
2017.09\textendash 2019.08 & \multicolumn{2}{l}{Killam Postdoc Fellow\hfill{}Department of Mathematics and Statistics,
Dalhousie University}\tabularnewline
2017.03\textendash 2017.08 & \multicolumn{2}{l}{Research Scientist, Johann Radon Inst. for Comp. and Appl. Math.,
Austrian Academy of Science}\tabularnewline
2016.06\textendash 2017.02 & \multicolumn{2}{l}{Postdoc Fellow\hfill{}Research Institute for Symbolic Computation,
Johannes Kepler University}\tabularnewline
 &  & \tabularnewline
\end{tabularx}%
\end{minipage}

\noindent{\large\textbf{\textsc{Research Interests}}}{\large\par}
\begin{center}
Symbolic Computation, Number Theory, Combinatorics, Special Functions
\begin{comment}
\noindent D. Li, H. Sun, C. Tao, and \textbf{L. Jiu}, Principal bundles
and holonomy groups on statistical manifolds, Submitted for Publication.

A. Dixit, X. Guan, \textbf{L. Jiu}, V. H. Moll and C. Vignat. The
integrals in Gradshteyn and Ryzhik. Part 32: the generating function
of Chebyshev polynomials of the second kind, To Appear in \emph{SCIENTIA
Series A: Mathematical Sciences.}

\textbf{L. Jiu} and V. H. Moll. The integrals in Gradshteyn and Ryzhik.
Part 31: rational functions, To Appear in \emph{SCIENTIA Series A:
Mathematical Sciences.}
\end{comment}
\par\end{center}

\noindent{\large\textbf{\textsc{Education}}}{\large\par}

\noindent{}%
\noindent\begin{minipage}[t]{1\columnwidth}%
\begin{tabularx}{\columnwidth}{l>{\raggedright\arraybackslash}Xr}
2011.08\textendash 2016.05 & Tulane University, Ph.D. in Mathematics & \emph{Advisor: }\emph{\uline{Victor H.~Moll}}\tabularnewline
\multirow{1}{*}{2013.09\textendash 2014.02} & Research Institute for Symbolic Computation, Johannes Kepler University,
Exchange Ph.D. Student & \emph{Advisor: }\emph{\uline{Carsten Schneider}}\tabularnewline
2008.09\textendash 2010.07 & Beijing Institute of Technology, Master of Science & \emph{Advisor: }\emph{\uline{Huafei Sun}}\tabularnewline
2004.09\textendash 2008.06 & Beijing Institute of Technology, Bachelor of Science & \emph{Thesis Advisor: }\emph{\uline{Huafei Sun}}\tabularnewline
 &  & \tabularnewline
\end{tabularx}%
\end{minipage}

\noindent{\large\textbf{\textsc{Grant Awarded}}}{\large\par}

\noindent{}%
\noindent\begin{minipage}[t]{1\columnwidth}%
\begin{tabularx}{\columnwidth}{l>{\raggedright\arraybackslash}X}
2023.07\textendash 2025.06 & \multicolumn{1}{l}{WHU-DKU Joint Grant Seed\hfill{}Wuhan University and Duke Kunshan
University}\tabularnewline
 & \multicolumn{1}{l}{DKU PI of ``Wuhan University-Duke Kunshan University-Dalhousie University
Research }\tabularnewline
 & \multicolumn{1}{l}{Platform on Combinatorics and Number Theory''}\tabularnewline
2023.01\textendash 2024.12 & \multicolumn{1}{l}{Faculty Learning Community, Center for Teaching and Learning, Duke
Kunshan University}\tabularnewline
2022.07\textendash 2024.06 & \multicolumn{1}{l}{WHU-DKU Joint Grant Seed\hfill{}Wuhan University and Duke Kunshan
University}\tabularnewline
 & \multicolumn{1}{l}{Research team member of Dr.~Dongmian Zou, Duke Kunshan University}\tabularnewline
2022.01\textendash 2022.12 & Gradescope Research Project Grant\hfill{}Gradescope\tabularnewline
 & Using Gradescope in math courses, facilitated by Center for Teaching
and Learning, Duke Kunshan University \tabularnewline
2021.07\textendash 2023.06 & Interdisciplinary Seed Grant\hfill{}Duke Kunshan University\tabularnewline
 & Quantum algorithms for computational number theory, linear algebra,
and combinatorics\tabularnewline
 & Joint with Dr.~Myung-Joong Huang, Duke Kunshan University\tabularnewline
2017.09\textendash 2019.08 & Killam Research Fund\hfill{}Killam Trust @ Dalhousie University\tabularnewline
 & Research Support for Killam Postdocs\tabularnewline
 & \tabularnewline
\end{tabularx}%
\end{minipage}

\noindent{\large\textbf{\textsc{Publications}}}\hspace{2cm}(While
working on the projects, undergraduate students are marked with a
{*})

\noindent\textsc{Book}

\noindent{}%
\noindent\begin{minipage}[t]{1\columnwidth}%
\begin{tabularx}{\columnwidth}{l>{\raggedright\arraybackslash}X}
1 & H.~Sun, L.~Peng, Y.~Cheng, D.~Li, and\textbf{ L.~Jiu}, \textsl{Mathematical
Foundations of Information Geometry}, Science Press, Beijing, 2025.
ISBN: 978-7-03-080107-4.\tabularnewline
 & \tabularnewline
\end{tabularx}%
\end{minipage}

\noindent\textsc{Papers}

\noindent{}%
\begin{xltabular}[l]{\columnwidth}{l>{\raggedright\arraybackslash}X}
39 & S.~Chern, \textbf{L.~Jiu}, S.~Li{*}, and L.~Wang, Leading coefficient
in the Hankel determinants related to binomial and $q$-binomial transforms,
submitted for publication.\tabularnewline
38 & \textbf{L.~Jiu} and D.~Wang{*}, On $b$-ary binomial coefficients
with negative entries, Submitted for Publication.\textbf{}\tabularnewline
37 & \textbf{L.~Jiu} and L.~Peng, Information geometry and alpha-parallel
prior of the beta-logistic distribution, \emph{Comm.~Statist.~Theory
Methods}. \textbf{54} (2025), 3292\textendash 3306. \tabularnewline
36 & S.~Chern, \textbf{L.~Jiu}, and I.~Simonelli, A central limit theorem
for a card shuffling problem, \emph{J.~Combin.~Theory Ser.~A} \textbf{214}
(2025), Article 106048. \tabularnewline
35 & \textbf{L.~Jiu} and Y.~Li{*}, Hankel determinants of certain sequences
of Bernoulli polynomials: A direct proof of an inverse matrix entry
from Statistics, \emph{Contrib.~Discrete Math}.~\textbf{19} (2024),
64\textendash 84.\tabularnewline
34 & Q.~Chen, S.~Chern, and \textbf{L.~Jiu}, Multi-headed lattices and
Green functions, \emph{J. Phys. A: Math. Theor}. \textbf{57} (2024)
Article 465204.\tabularnewline
33 & S.~Chern and \textbf{L.~Jiu}, Hankel determinants and Jacobi continued
fractions for $q$-Euler numbers, \emph{C.~R.~Math. Acad.~Sci.~Paris
}\textbf{362} (2024), 203\textendash 216.\tabularnewline
32 & K.~Dilcher and \textbf{L.~Jiu}, Hankel determinants of shifted sequences
of Bernoulli and Euler numbers, \emph{Contrib.~Discrete Math}.~\textbf{18}
(2023), 146\textendash 175.\tabularnewline
31 & Z.~Bradshaw, I.~Gonzalez, \textbf{L.~Jiu}, V.~H.~Moll, and C.~Vignat,
Compatibility of the method of brackets with classical integration
rules, \emph{Open Math}.~\textbf{21} (2023), Article number: 20220581.\tabularnewline
30 & \textbf{L.~Jiu} and D.~Y.~H.~Shi, Moments and cumulants on identities
for Bernoulli and Euler numbers, \emph{Math. Reports }\textbf{24}
(2022), 643\textendash 650.\tabularnewline
29 & \textbf{L.~Jiu} I.~Simonelli, and H.~Yue{*}, Loop Decompositions
of Random Walks and Nontrivial Identities of Bernoulli and Euler Polynomials,
\emph{Integers} \textbf{22} (2022), A91.\tabularnewline
28 & K.~Dilcher and \textbf{L.~Jiu}, Hankel Determinants of sequences
related to Bernoulli and Euler Polynomials, \emph{Int.~J.~Number
Theory }\textbf{18} (2022), 331\textendash 359.\tabularnewline
27 & K.~Dilcher and \textbf{L.~Jiu}, Orthogonal polynomials and Hankel
determinants for certain Bernoulli and Euler polynomials, \emph{J.~Math.~Anal.~Appl}.~\textbf{497}
(2021), Article 124855.\tabularnewline
26 & I.~Gonzales, \textbf{L.~Jiu}, and V.~H.~Moll, An extension of
the method of brackets. Part 2, \emph{Open Math}.~\textbf{18} (2020),
983\textendash 955.\tabularnewline
25 & \textbf{L.~Jiu} and C.~Koutschan, Calculation and properties of
zonal polynomials,\emph{ Math.~Comput.~Sci}.~\textbf{14} (2020),
623\textendash 640.\tabularnewline
24 & N.~Takayama, \textbf{L.~Jiu}, S.~Kuriki, and Y.~Zhang, Computations
of the Expected Euler Characteristic for the Largest Eigenvalue of
a Real Wishart Matrix, \emph{J.~Multivariate Anal.~}\textbf{179}
(2020), Article 104642.\tabularnewline
23 & \textbf{L.~Jiu}, C.~Vignat, and T.~Wakhare, Analytic Continuation
for Multiple Zeta Values using Symbolic Representations, \emph{Int.~J.~Number
Theory} \textbf{16} (2020), 579\textendash 602.\tabularnewline
22 & \textbf{L.~Jiu} and C.~Vignat, Connection coefficients for higher-order
Bernoulli and Euler polynomials: a random walk approach, \emph{Fibonacci
Quart.~}\textbf{57} (2019), 84\textendash 95.\tabularnewline
21 & \textbf{L.~Jiu} and D.~Y.~H.~Shi, Matrix representation for multiplicative
nested sums, \emph{Colloq.~Math}.\textbf{~158} (2019), 183\textendash 194.\tabularnewline
20 & \textbf{L.~Jiu} and D.~Y.~H.~Shi, Orthogonal polynomials and connection
to generalized Motzkin numbers for higher-order Euler polynomials,
\emph{J.~Number Theory}�\textbf{199} (2019), 389\textendash 402.\tabularnewline
19 & I.~Gonzalez, K.~Kohl, \textbf{L.~Jiu}, and V.~H.~Moll, The method
of brackets in experimental mathematics, \emph{Frontiers of Orthogonal
Polynomials and q-Series}, Z.�Nashed and X.�Li eds., World Scientific
Publishers, 2018.\tabularnewline
18 & \textbf{L.~Jiu}, V.~H.~Moll, and C.~Vignat, A symbolic approach
to multiple zeta values at the negative integers,\emph{ J.�~Symbolic
Comput}.~\textbf{84} (2018), 1\textendash 13.\tabularnewline
17 & I.~Gonzales, K.~Kohl, \textbf{L.~Jiu}, and V.~H.~Moll, An extension
of the method of brackets. Part 1, \emph{Open Math}.~\textbf{15}
(2017), 1181\textendash 1211.\tabularnewline
16 & \textbf{L.~Jiu}, Integral representations of equally positive integer-indexed
harmonic sums at infinity, \emph{Research in Number Theory} \textbf{3}
(2017), Article 3:10.\tabularnewline
15 & C.~Li, E.~Zhang, \textbf{L.~Jiu}, and H.~Sun, Optimal control
on special Euclidean group via natural gradient descent algorithm,
\emph{Sci.~China Inf.~Sci}.~\textbf{59} (2016), Article: 112203.\tabularnewline
14 & I.~Gonzalez, \textbf{L.~Jiu}, and V.~H.~Moll, Pochhammer symbol
with negative indices. A new rule for the method of brackets, \emph{Open
Math}.~\textbf{14} (2016), 681\textendash 686.\tabularnewline
13 & T.~Amdeberhan, A.~Dixit, X.~Guan, \textbf{L.~Jiu}, A.~Kuznetsov,
V.~H.~Moll, and C.~Vignat, The integrals in Gradshteyn and Ryzhik.
Part 30: trigonometric functions, \emph{Scientia Series A: Mathematical
Sciences} \textbf{27} (2016), 47\textendash 74.\tabularnewline
12 & T.~Amdeberhan, A.~Dixit, X.~Guan, \textbf{L.~Jiu}, V.~H.~Moll,
and C.~Vignat, A series involving Catalan numbers. Proofs and demonstrations,
\emph{Elem.~Math}.~\textbf{71} (2016), 109\textendash 121.\tabularnewline
11 & \textbf{L.~Jiu} and C.~Vignat, On binomial identities in arbitrary
bases, \emph{J.~Integer Seq}.~\textbf{19} (2016), Article 16.5.5.\tabularnewline
10 & \textbf{L.~Jiu}, V.~H.~Moll, and C.~Vignat, A symbolic approach
to some identities for Bernoulli-Barnes polynomials, \emph{Int.~J.~Number
Theory} \textbf{12} (2016), 649\textendash 662.\tabularnewline
9 & A.~Dixit, \textbf{L.~Jiu}, V.~H.~Moll, and C.~Vignat, The finite
Fourier transform of classical polynomials, \emph{J.~Aust. Math.~Soc}.\textbf{~98}
(2015), 145\textendash 160.\tabularnewline
8 & T.~Amdeberhan, A.~Dixit, X.~Guan, \textbf{L.~Jiu} and V.~H.~Moll,
The unimodality of a polynomial coming from a rational integral. Back
to the original proof, \emph{J.~Math.~Anal.~Appl}.~\textbf{420}
(2014), 1154\textendash 1166.\tabularnewline
7 & A.~Byrnes{*}, \textbf{L.~Jiu}, V.�H.�Moll, and C.�Vignat, Recursion
rules for the hypergeometric zeta functions, \emph{Int.~J.~Number
Theory} \textbf{10} (2014), 1761\textendash 1782.\tabularnewline
6 & \textbf{L.~Jiu}, V.~H.~Moll, and C.~Vignat, Identities for generalized
Euler polynomials, Integral Transforms \emph{Spec. Funct}.~\textbf{25}
(2014), 777\textendash 789.\tabularnewline
5 & Z.~Zhang, H.~Sun, \textbf{L.~Jiu}, and L.~Peng, A natural gradient
algorithm for stochastic distribution systems, \emph{Entropy} \textbf{16}
(2014), 4338\textendash 4352.\tabularnewline
4 & F.~Zhang, H.~Sun, \textbf{L.~Jiu}, and L.~Peng, The arc length
variational formula on the exponential manifold, \emph{Math.~Slovaca}
\textbf{63} (2013), 1101\textendash 1112.\tabularnewline
3 & L.~Peng, H.~Sun, and \textbf{L.~Jiu}, The geometric structure of
the Pareto distribution, \emph{Bol.~Asoc.~Mat.~Venez}.~\textbf{14}
(2007), 5\textendash 13.\tabularnewline
2 & \textbf{L.~Jiu} and H.~Sun, On minimal homothetical hypersurfaces,
\emph{Colloq.~Math}.~109 (2007), 239\textendash 249.\tabularnewline
1 & X.~Wang and \textbf{L.~Jiu}, Characterizing hypersurfaces of generalized
rotation through its normal lines, \emph{Journal of Ningde Normal
University (Natural Science)} \textbf{02} (2006), 117\textendash 119.\tabularnewline
 & \tabularnewline
\end{xltabular}

\noindent{\large\textbf{\textsc{Academic Talks}}}{\large\par}

\noindent{}%
\begin{xltabular}[l]{\columnwidth}{l>{\raggedright\arraybackslash}X}
47 & \textbf{Examples of Computer Proofs: From Elementary to Recent Ones}\tabularnewline
 & \emph{Invited Honours Seminar Talk, }Department of Mathematics and
Statistics, Dalhousie University, Halifax, NS, Canada, Jan.~15, 2025.\tabularnewline
46 & \textbf{Multi-headed Lattices and Green Functions}\tabularnewline
 & \emph{Invited Seminar Talk, }Department of Mathematics and Statistics,
Dalhousie University, Halifax, NS, Canada, Oct.~8, 2024.\tabularnewline
45 & \textbf{$q$-Analogue on Hankel Determinants: the $q$-Euler Numbers
and the $q$-Binomial Transform}\tabularnewline
 & \emph{Canadian Number Theory Association XVI, }Fields Institute, Toronto,
ON, Canada, June 10\textendash 14, 2024.\tabularnewline
44 & \textbf{Shuffle to One, Shuffle to Normal}\tabularnewline
 & \emph{Number Theory Seminar, Department of Mathematics and Statistics,
Dalhousie University}, Halifax, NS, Canada, Jan.~31, 2024.\tabularnewline
43 & \textbf{Random Walk Models for Identities Involving Bernoulli and
Euler Polynomials}\tabularnewline
 & \emph{Number Theory Seminar, Department of Mathematics and Statistics,
Dalhousie University}, Halifax, NS, Canada, Mar.~6, 2023.\tabularnewline
42 & \textbf{Random Walk Model on Finite Number of Sites}\tabularnewline
 & \emph{Seminar}\textsl{, School of Mathematics, Anhui University},
Online, Oct.~19, 2022.\tabularnewline
41 & \textbf{Bernoulli Symbol and Multiple Zeta Function at Non-negative
Integers}\tabularnewline
 & \emph{The First International Conference on Multiple Zeta Values and
Related Topics}, Online, Aug.~08\textendash 09, 2022.\tabularnewline
40 & \textbf{Hankel Determinants of Certain Sequences of Bernoulli and
Euler Polynomials}\tabularnewline
 & \emph{Seminar}\textsl{, Department of Mathematics, Zhejiang Sci-Tech
University}, Online, June 12, 2022.\tabularnewline
39 & \textbf{Bernoulli and Euler Symbols: Umbral Calculus, Random Variables,
and Multiple Zeta Values}\tabularnewline
 & \emph{Duke Kunshan University-Shanghai Jiao Tong University Joint
Workshop for Mathematics and Data Science}, Shanghai, P.~R.~China,
Jan.~5, 2022.\tabularnewline
38 & \textbf{Random Walk Models for Non-trivial Identities Involving Bernoulli
and Euler Polynomials of Higher-orders}\tabularnewline
 & \emph{Suzhou Area Youth Mathematicians 2nd Annual Workshop,} Soochow
University, Kunshan, Suzhou, Jiangsu Province, P.~R.~China, Sept.~25\textendash 26,
2021.\tabularnewline
37 & \textbf{Random Walks and Identities Involving Bernoulli and Euler
Polynomials of Higher-order}\tabularnewline
 & \textsl{Seminar, Institute of Statistics and Big Data, Renmin University
of China}, Beijing, P. R. China, June 18, 2021.\tabularnewline
36 & \textbf{Examples on Computer Proofs}\tabularnewline
 & \emph{Seminar}, \textsl{Wuhan University}, Wuhan, Hubei Province,
P. R. China, May 28, 2021.\tabularnewline
35 & \textbf{Hankel Determinant of Sequences Related to Bernoulli and Euler
Polynomials}\tabularnewline
 & \emph{DKU-WHU Math and Stat Academic Conference}, Wuhan University,
Wuhan, Hubei Province, P. R. China, May 27, 2021.\tabularnewline
34 & \textbf{Hankel Determinant on Sequences Related to Bernoulli and Euler
Polynomials}\tabularnewline
 & \emph{Suzhou Area Youth Mathematicians 1st Annual Workshop,} Duke
Kunshan University, Kunshan, Suzhou, Jiangsu Province, P.~R.~China,
Nov.~14\textendash 15, 2020.\tabularnewline
33 & \textbf{Three Examples on Computer Proofs}\tabularnewline
 & \emph{Zu Chongzhi Colloquium Series}\textsl{\emph{,}}\textsl{ Duke
Kunshan University}, Kunshan, Suzhou, P.~R.~China, Nov.~6, 2020.\tabularnewline
32 & \textbf{Introduction to Four Symbolic Integration Methods: Two Examples}\tabularnewline
 & \emph{Number Theory Seminar, Department of Mathematics and Statistics,
Dalhousie University}, Halifax, NS, Canada, Sept. 23, 2019\tabularnewline
31 & \textbf{On $b$-ary Binomial Coefficients}\tabularnewline
 & \emph{Number Theory Seminar, Department of Mathematics and Statistics,
Dalhousie University}, Halifax, NS, Canada, Sept. 16, 2019\tabularnewline
30 & \textbf{Orthogonal Polynomials for Higher-order Euler Polynomials}\tabularnewline
 & \emph{15th International Symposium on Orthogonal Polynomials, Special
Functions and Applications}, Hagenberg, Austria, July 22\textendash 26,
2019.\tabularnewline
29 & \textbf{On Harmonic Sums: Integral and Matrix Representations with
Connections to Partition-theoretic Generalization of the Riemann Zeta-function
and Random Walks}\tabularnewline
 & \emph{Analytic and Combinatorial Number Theory: The Legacy of Ramanujan
(A conference in honor of Bruce C.~Berndt's 80th birthday)}, University
of Illinois at Urbana-Champaign, Urbana, IL, U.~S.~A., June 6\textendash 9,
2019.\tabularnewline
28 & \textbf{Random Walk Approaches to Identities on Higher-order Bernoulli
and Euler Polynomials}\tabularnewline
 & \emph{American Mathematical Society Spring Southeastern Sectional
Meeting}, Auburn University, Auburn, AL, U.~S.~A., Mar.~15\textendash 17,
2019.\tabularnewline
27 & \textbf{Random Walk \& Identities}\tabularnewline
 & \emph{Number Theory Seminar, Department of Mathematics and Statistics,
Dalhousie University}, Halifax, NS, Canada, Feb. 25, 2019\tabularnewline
26 & \textbf{Matrix Representation for Multiplicative Nested Sums}\tabularnewline
 & \emph{2019 Joint Mathematics Meetings}, Baltimore, MD, U.~S.~A.,
Jan.~16\textendash 19, 2019.\tabularnewline
25 & \textbf{Orthogonal Polynomials for Bernoulli and Euler Polynomials}\tabularnewline
 & \emph{Number Theory Seminar, Department of Mathematics and Statistics,
Dalhousie University}, Halifax, NS, Canada, Jan. 7, 2019\tabularnewline
24 & \textbf{Three Examples of Computer Proofs of Combinatorial Results}\tabularnewline
 & \emph{Honours Seminar}\textsl{\emph{, }}\textsl{Department of Mathematics
and Statistics, Dalhousie University}, Halifax, NS, Canada, Oct. 17,
2018\tabularnewline
23 & \textbf{Matrix Representation for Multiplicative Nested Sums}\tabularnewline
 & \emph{Number Theory Seminar, Department of Mathematics and Statistics,
Dalhousie University}, Halifax, NS, Canada, Sept. 21, 2018.\tabularnewline
22 & \textbf{Bernoulli Symbol and Sum of Powers}\tabularnewline
 & \emph{6th International Congress on Mathematical Software}, University
of Notre Dame, Notre Dame, IN, U.~S.~A., July 24\textendash 27,
2018.\tabularnewline
21 & \textbf{Random Walks and Identities for High-order Bernoulli and Euler
Polynomials}\tabularnewline
 & \emph{18th International Conference on Fibonacci Numbers and Their
Applications}, Dalhousie University, Halifax, NS, Canada, July 1\textendash 8,
2018.\tabularnewline
20 & \textbf{Matrix Representations for Bernoulli and Euler Polynomials}\tabularnewline
 & \emph{2018 Canadian Mathematical Society Summer Meeting}, University
of New Brunswick, Fredericton, NB, Canada, June 1\textendash 4, 2018.\tabularnewline
19 & \textbf{Two Sequences Related to Bernoulli and Euler Numbers}\tabularnewline
 & \emph{Number Theory Seminar, Department of Mathematics and Statistics,
Dalhousie University}, Halifax, NS, Canada, May 30, 2018.\tabularnewline
18 & \textbf{Hidden Walks}\tabularnewline
 & \emph{Number Theory Seminar, Department of Mathematics and Statistics,
Dalhousie University}, Halifax, NS, Canada, Feb. 26, 2018.\tabularnewline
17 & \textbf{Introduction to Zonal Polynomials}\tabularnewline
 & \emph{Number Theory Seminar, Department of Mathematics and Statistics,
Dalhousie University}, Halifax, NS, Canada, Jan. 22, 2018.\tabularnewline
16 & \textbf{The Probabilistic and Combinatorial Interpretations of the
Bernoulli Symbol}\tabularnewline
 & \emph{2017 Canadian Mathematical Society Winter Meeting}, University
of Waterloo, Waterloo, ON, Canada, Dec. 8\textendash 11, 2017.\tabularnewline
15 & \textbf{Bernoulli Symbol on Multiple Zeta Values at Negative Integers}\tabularnewline
 & \emph{23rd Conference on Applications of Computer Algebra (Commemorating
the heritage of Jonathan Michael Borwein)}, Jerusalem College of Technology,
Jerusalem, Israel, July 17\textendash 21, 2017.\tabularnewline
14 & \textbf{Bernoulli Symbol $\mathcal{B}$: from Umbral Calculus to Random
Variable and Combinatorics}\tabularnewline
 & \emph{Number Theory Seminar, Department of Mathematics and Statistics,
Dalhousie University}, Halifax, NS, Canada, Oct. 13, 2017.\tabularnewline
13 & \textbf{Visualization of Bernoulli Numbers}\tabularnewline
 & \emph{Colloquium, Department of Mathematics and Statistics, Dalhousie
University}, Halifax, NS, Canada, Oct. 12, 2017.\tabularnewline
12 & \textbf{On Bernoulli Symbol $\mathcal{B}$}\tabularnewline
 & \emph{Klagenfurt-Linz-Wien Workshop}, Riefnitz, Austria, May 3\textendash 6,
2017.\tabularnewline
11 & \textbf{The Method of Brackets (MoB) and Integrating by Differentiating
(IbD) Method}\tabularnewline
 & \emph{Laboratoire des Signaux et System}\`{e}\emph{s, Universit}{\'{e}}\emph{
Paris Sud XI}, Orsay, France, Dec.~9, 2016.\tabularnewline
10 & \textbf{``Random Walks'' for Harmonic Sums}\tabularnewline
 & \emph{SFB Statusseminar}, Strobl, Austria, Nov.~27\textendash 30,
2016.\tabularnewline
9 & \textbf{A Hot Pot}\tabularnewline
 & \textsl{Algorithmic Combinatorics Seminar, Research Institute for
Symbolic Computations, Johannes Kepler University}, Hagenberg im M\"{u}hlkreis,
Austria, Oct. 5, 2016.\tabularnewline
8 & \textbf{On Binomial Identities in Arbitrary Bases}\tabularnewline
 & \emph{Beijing Key Laboratory on Mathematical Characterization, Analysis
and Applications of Complex Information, Beijing Institute of Technology},
Beijing, China, July 26, 2016.\tabularnewline
7 & \textbf{Random Walk: A Probabilistic and Geometric Approach to Number
Theory}\tabularnewline
 & \emph{International Conference on Mathematical Characterization, Analysis
and Applications of Complex Information}, Beijing Institute of Technology,
Beijing, China, July 19\textendash 20, 2016.\tabularnewline
6 & \textbf{The Method of Brackets}\tabularnewline
 & \emph{5th International Congress on Mathematical Software}, The Zuse
Institute Berlin, Berlin, Germany, July 11\textendash 14, 2016.\tabularnewline
5 & \textbf{The Method of Brackets}\tabularnewline
 & \textsl{Algorithmic Combinatorics Seminar, Research Institute for
Symbolic Computations, Johannes Kepler University}, Hagenberg im M\"{u}hlkreis,
Austria, June 22, 2016.\tabularnewline
4 & \textbf{Binomial Identities in Arbitrary Bases}\tabularnewline
 & \textsl{Graduate Students Colloquium, Department of Mathematics, Tulane
University}, New Orleans, LA., U.~S.~A., Mar. 8, 2016\tabularnewline
3 & \textbf{On Bernoulli Symbol $\mathcal{B}$ and Its Applications}\tabularnewline
 & \emph{Center for Combinatorics, Nankai University}, Tianjin, China,
July 8, 2015.\tabularnewline
2 & \textbf{Recursion Rules for the Hypergeometric Zeta Functions}\tabularnewline
 & \emph{Midwest Number Theory Conference for Graduate Students and Recent
PhDs, X,} University of Illinois at Urbana-Champaign, Urbana, IL,
U.~S.~A., June 3\textendash 4, 2014.\tabularnewline
1 & \textbf{Implementation of an Algorithm on Converting Sums into Nested
Sums}\tabularnewline
 & \emph{Laboratoire des Signaux et Systemes, Universit}{\'{e}}\emph{
Paris Sud XI}, Orsay, France, Jan.~8, 2014.\tabularnewline
 & \tabularnewline
\end{xltabular}

\noindent{\large\textbf{\noun{Honors and Awards}}}{\large\par}

\noindent{}%
\noindent\begin{minipage}[t]{1\columnwidth}%
\noindent{}%
\begin{longtable}[c]{l>{\raggedright}p{0.38\paperwidth}>{\raggedleft}p{0.28\paperwidth}}
2016 & Tea Doctor (for organizing departmental Tea Time) & Depart. of Math., Tulane University\tabularnewline
2015 & Tea Master (for organizing departmental Tea Time) & Depart. of Math., Tulane University\tabularnewline
2014 & Excellence in Mathematics & Depart. of Math., Tulane University\tabularnewline
2013 & Excellent Graduate Student Teacher & Depart. of Math., Tulane University\tabularnewline
2008 & Outstanding Graduates & Beijing Institute of Technology\tabularnewline
2007 & National Scholarship & Department of Education, P.~R.~China\tabularnewline
2006 & China Aerospace Science and Technology Corporation Scholarship, 2nd
Prize & China Aerospace Science and Technology Corporation\tabularnewline
 &  & \tabularnewline
\end{longtable}%
\end{minipage}

\noindent{\large\textbf{\noun{Teaching Experience}}}{\large\par}

\noindent{}%
\begin{xltabular}[c]{\columnwidth}{ll>{\raggedright\arraybackslash}Xr}
2025 Winter & MATH 6400 & Integer Partitions and $q$-Series & Dalhousie University\tabularnewline
2024 Fall & MATH 307 & Complex Analysis & Duke Kunshan University\tabularnewline
2023 Fall & MATH 105 & Calculus & Duke Kunshan University\tabularnewline
 & MATH 202 & Linear Algebra & Duke Kunshan University\tabularnewline
 & MATH 105 & Calculus & Duke Kunshan University\tabularnewline
 & MATH 301 & Advanced Introduction to Probability & Duke Kunshan University\tabularnewline
2023 Spring & MATH 205 & Probability and Statistics & Duke Kunshan University\tabularnewline
 & MINITERM 102 & {\small Experimental Mathematics and Symbolic Computation} & Duke Kunshan University\tabularnewline
2022 Fall & INDSTU 391 & Introduction to Algebraic Geometry & Duke Kunshan University\tabularnewline
 & MATH 105 & Calculus & Duke Kunshan University\tabularnewline
 & MATH 306 & Number Theory & Duke Kunshan University\tabularnewline
 & MATH 301 & Advanced Introduction to Probability & Duke Kunshan University\tabularnewline
2022 Spring & INDSTU 391 & Variational Quantum Algorithms & Duke Kunshan University\tabularnewline
 & MATH 201 & Multivariable Calculus & Duke Kunshan University\tabularnewline
 & MATH 301 & Advanced Introduction to Probability & Duke Kunshan University\tabularnewline
 & MATH 201 & Multivariable Calculus & Duke Kunshan University\tabularnewline
2021 Fall & MATH 105 & Calculus & Duke Kunshan University\tabularnewline
 & INDSTU 391 & Riemann Zeta-Function & Duke Kunshan University\tabularnewline
 & INDSTU 391 & Quantum Algorithm & Duke Kunshan University\tabularnewline
 & MATH 306 & Number Theory & Duke Kunshan University\tabularnewline
 & INDSTU 391 & Combinatorics & Duke Kunshan University\tabularnewline
2021 Spring & MATH 205 & Probability and Statistics & Duke Kunshan University\tabularnewline
 & MATH 301 & Advanced Introduction to Probability & Duke Kunshan University\tabularnewline
2020 Fall & MATH 105 & Calculus & Duke Kunshan University\tabularnewline
 & MATH 201 & Multivariable Calculus & Duke Kunshan University\tabularnewline
2019 Summer & MATH 1030 & Matrix Theory and Linear Algebra I & Dalhousie University\tabularnewline
2019 Winter & MATH 3080 & Introduction to Complex Variables & Dalhousie University\tabularnewline
2016 Spring & MATH 1060 & Long Calculus II & Tulane University\tabularnewline
2015 Fall & MATH 1310 & Consolidated Calculus & Tulane University\tabularnewline
2015 Spring & MATH 1210 & Long Calculus I & Tulane University\tabularnewline
2014 Summer & MATH 1160 & Long Calculus II & Tulane University\tabularnewline
\end{xltabular}{\large\textbf{\textsc{Academic Services and Memberships}}}{\large\par}

\noindent{}%
\noindent\begin{minipage}[t]{1\columnwidth}%
\begin{tabularx}{\columnwidth}{l>{\raggedright\arraybackslash}X}
Since 2025 & Reviewer for Mathematical Reviews\tabularnewline
2025 & Organizing the mini-symposium \textsl{Special Functions with Applications
in Number Theory and Combinatorics} at The Third Joint SIAM/CAIMS
Annual Meetings\tabularnewline
 & \tabularnewline
\end{tabularx}%
\end{minipage}

\noindent{\large\textbf{\textsc{Departmental and University Service}}}{\large\par}

\noindent{}%
\noindent\begin{minipage}[t]{1\columnwidth}%
\begin{tabularx}{\columnwidth}{l>{\raggedright\arraybackslash}X}
2021\textendash{} & Organizer of the Discrete Math Seminar\hfill{}Duke Kunshan University\tabularnewline
2024 & Member of 2025 Undergraduate Recruitment \& Admissions Evaluation\hfill{}Duke
Kunshan University\tabularnewline
2017\textendash 2020 & Organizer of the Number Theory Seminar\hfill{}Dalhousie University\tabularnewline
2012\textendash 2016 & Organizer of the Tee Time\hfill{}Tulane University\tabularnewline
\end{tabularx}%
\end{minipage}

\noindent{\large\textbf{\textsc{Student Mentoring}}}{\large\par}

\noindent\textsc{Undergraduate Academic Advisor @ Duke Kunshan University}

\vspace{-20bp}
\noindent{}%
\noindent\begin{minipage}[t]{1\columnwidth}%
\noindent{}%
\begin{xltabular}[c]{\columnwidth}{ll>{\raggedright\arraybackslash}X}
\textit{Class of} & \textit{Number of Students} & \textit{Names}\tabularnewline
2022 & 1 & Ziang Zhou\tabularnewline
2023 & 6 & Heng Yue, Junyu Shi, Lezhen Qin, Mengfan Gong, Yushan Gu, Shi Wang\tabularnewline
2024 & 3 & Jeff Ulmasov, Jing Gu, Yuekang Li\tabularnewline
2025 & 1 & Jiaqi Wang\tabularnewline
2026 & 3 & Baoguanyan Kang, Dalia Guerrero Flores, Lei Wu,\tabularnewline
2027 & 4 & Jinggege Li, Rui Ling, Shengjie Bai, Yunjie Guo\tabularnewline
2028 & 5 & Binghan Cheng, Feiyang Zhong, Rustam Safaev, Shengyu Xu, Yihang Yin\tabularnewline
TOTAL & 23 & \tabularnewline
\end{xltabular}%
\end{minipage}

\noindent\textsc{Undergraduate Signature Work (SW)}\footnote{https://signature-work.dukekunshan.edu.cn/signature-work-overview/}\textsc{
($\cong$ Honor Thesis) Mentor @ Duke Kunshan University}\vspace{-20bp}

\noindent{}%
\noindent\begin{minipage}[t]{1\columnwidth}%
\noindent{}%
\begin{xltabular}[c]{\columnwidth}{ll>{\raggedright\arraybackslash}X}
\textit{Class of} & \textit{Number of Projects} & \textit{Names}\tabularnewline
2023 & 7 & Heng Yue, Junyu Shi, Lezhen Qin, Mengfan Gong, Siyuan Wu, Ye Li, Youzhang
He\tabularnewline
2024 & 3 & Hongkai Zhu, Matilde Molinari Giglietti, Shuhan Li\tabularnewline
TOTAL & 10 & \tabularnewline
\end{xltabular}%
\end{minipage}

\noindent\textsc{Undergraduate Student Research Projects}\vspace{-20bp}

\noindent{}%
\noindent\begin{minipage}[t]{1\columnwidth}%
\noindent{}%
\begin{xltabular}[c]{\columnwidth}{lll>{\raggedright\arraybackslash}X}
\textit{Year/Term} & \textit{Student(s)} & \textit{Project(s)} & \textit{Result/Comments}\tabularnewline
\midrule
\multirow{2}{*}{2021 Summer} & Heng Yue & Loop Decomposition of Random Walks & {[}29{]} in the Publication section\tabularnewline
 & Ye Li & Hankel Determinants on Some sequences & {[}35{]} in the Publication section\tabularnewline
\midrule
\multirow{2}{*}{2022 Summer} & Siyuan Wu & The Method of Brackets & Methematica Package\tabularnewline
 & Duanduan Wang & $b$-ary Related Sequences & {[}38{]} in the Publication section\tabularnewline
\midrule
\multirow{2}{*}{2023 Summer} & Shuhan Li & Hankel Determinants and Continued Fractions & {[}39{]} in the Publication section\tabularnewline
 & Hongkai Zhu & Weakly Increasing Trees on a Multiset & Mathematica Package for SW\tabularnewline
\midrule
\multirow{2}{*}{2024 Summer} & Jonah Barrington & On Cyclotomic Polynomials & \multirow{2}{=}{Co-mentored with K.~Dilcher @ Dalhousie University}\tabularnewline
 & Julius Frizzell & Factorization of Large Primes & \tabularnewline
 &  &  & \tabularnewline
\end{xltabular}%
\end{minipage}

\noindent{\large\textbf{\textsc{Relevant Skills}}}{\large\par}

\noindent{}%
\noindent\begin{minipage}[t]{1\columnwidth}%
\begin{tabular}{lll}
\textsl{Language: } & \multicolumn{2}{l}{Mandarin (native), English (fluent)}\tabularnewline
\multirow{1}{*}{\emph{Computer:}} & \multicolumn{2}{l}{Mathematica, SageMath, Python, Maple, \LaTeX, \LyX}\tabularnewline
\emph{Packages:} & Zonal.sage & \url{https://jiulin90.github.io/Packages/Zonal.sage}\tabularnewline
 & BNE.sage & \url{https://jiulin90.github.io/Packages/BNE.sage}\tabularnewline
\end{tabular}%
\end{minipage}
\end{document}
